\documentclass[DIV=13]{scrartcl}
\usepackage{ulGenStyles}
\usepackage{lilyglyphsStyle}


%%%%%%%%%%%%%%%%%%%%
%% Hauptschriften %%
\setmainfont[%
	Numbers=OldStyle,
	Numbers=Proportional]	
	{Minion Pro}
\setsansfont[%
	Numbers=OldStyle,
	Numbers=Proportional]
	{Cronos Pro}


\pagestyle{empty}
\begin{document}

\section*{\centering\Huge\lilyglyphs[scale=1.3]}
\begin{flushright}
Urs Liska, November 2012
\end{flushright}
Lange Zeit hatte ich mich um das Problem herumlaviert, Notenzeichen in Texten verwenden zu müssen.
Aus Anlass eines Revisionsberichts zu einer Notenausgabe habe ich mich der Fragestellung schließlich noch einmal von Grund auf angenommen und eine elegante und umfassende Lösung gefunden: 
die Entwicklung von \texttt{lilyglyphs}, eines Pakets, das die Verwendung beliebiger Notationselemente im Fließtext von \LaTeX-Dokumenten ermöglicht.

\LaTeX ist ein sehr ausgreiftes Textsatzsystem, dessen wesentlicher Zweck es ist, Dokumente in professioneller Satzqualität zu erzeugen%
\footnote{\url{http://de.wikipedia.org/wiki/LaTeX}}%
.
Im Gegensatz zu Textverarbeitungsprogrammen basiert die Arbeit mit \LaTeX{} auf Dateien im Reintext-Format (sog. „Quelltexten“), in denen Formatierungen, Sonderzeichen, Tabellen etc. nicht grafisch eingefügt und bearbeitet, sondern durch „Befehle“ definiert und konfiguriert werden.
\LaTeX{} „kompiliert“ die Quelldateien schließlich zu druckbaren PDF-Dokumenten. 
In zahlreichen wissenschaftlichen Disziplinen ist der Umgang mit \LaTeX{} Standard, leider jedoch nicht in den Geistes- und insbesondere Musikwissenschaften.

\texttt{liylglyphs} erlaubt die Einbindung prinzipiell aller Notationselemente, die mit dem Notensatzprogramm LilyPond\footnote{\url{http://www.lilypond.org}} erzeugt werden können, als simple \LaTeX-Kommandos.
LilyPond ist ein -- ebenfalls Text\-datei-basiertes Noten\emph{satz}-Programm, dessen Stärke u.\,a. im „schweren“, „handgestochenen“ Charakter der Partituren liegt.
Ein wesentlicher Grund für diesen Eindruck ist der Notensatz-Font der Software.
\texttt{lilyglyphs} erlaubt nun die direkte Einbindung sämtlicher Zeichen dieses Fonts in \LaTeX-Doku\-mente.
Wie man beispielsweise das Copyright-Symbol \copyright{} durch den Standardbefehl \cmd{copyright} erzeugen würde, kann man nun eine Fermate \fermata durch den Befehl \cmd{fermata} einfügen.
Symbole, die nicht als Zeichen im Font vorliegen -- wie etwa Noten, die von LilyPond aus mehreren Elementen zusammengefügt werden --, können über den Umweg kleiner \textsc{pdf}-Bilddateien eingefügt werden.
In der Verwendung im Dokument ergibt sich allerdings kein Unterschied, eine Viertelnote \crotchet[scale=1.1] wird durch ein simples \cmd{crotchet} erzeugt.

Für die Verwendung von \texttt{lilyglyphs} sind weder eine Installation von LilyPond noch Kenntnisse im Umgang damit erforderlich.
Beides wird nur für die Erzeugung neuer bild-basierter Kommandos gebraucht.
Umgekehrt kann man dafür -- dank der weitgehenden Automatisierung durch ein generierendes Skript auf einfachste Weise -- Kommandos aus jeglichen mit LilyPond setzbaren Konstrukten erstellen.
Sämtliche Elemente aus dem eigentlichen Zeichensatz sind dagegen unmittelbar über ihren Namen zugänglich, auch wenn sie noch nicht als vordefinierte Befehle zur Verfügung stehen, wie beispielsweise der Mensur-Schlüssel \lilyGlyph[scale=1, raise=0.75]{clefs.mensural.c} durch \cmd{lilyGlyph\{clefs.mensural.c\}}.
Diese leichte Erweiterbarkeit um beliebige Elemente unterscheidet \texttt{lilyglyphs} wesentlich von anderen mir bekannten Ansätzen.
Ein weiterer Pluspunkt ist die automatische Skalierung der Zeichen mit der aktuellen Schriftgröße.

\medskip
Die Projekt-Homepage von \texttt{lilyglyphs} ist \url{https://github.com/lilyglyphs/lilyglyphs}. 
Dort ist das Paket und eine ausführliche (allerdings nur Englische) Dokumentation zum Download erhältlich.
\texttt{lilyglyphs} ist Freie Software im Sinne der GNU General Public License, es fallen also keine Kosten für die Verwendung an.
Mitarbeit in Form von Diskussion, Fehlerberichten, Verbreitung oder Beiträgen zur Ergänzung der Bibliothek wird dagegen sehr gerne gesehen.

Für weitergehende Fragen, auch zum Potenzial von Textdatei-basierten Arbeitsprozessen, stehe ich unter der Mail-Adresse \href{mailto:mail@ursliska.de}{mail@ursliska.de} gerne zur Verfügung.

\end{document}