%%%%%%%%%%%%%%%%%%%%%%%%%%%%%%%%%%%%%%%%%%%%%%%%%%%%%%%%%%%%%%%%%%%%%%%%%%
%                                                                        %
%      This file is part of the 'lilyglyphs' LaTeX package.              %
%                                ==========                              %
%                                                                        %
%              https://github.com/openlilylib/lilyglyphs                 %
%               http://www.openlilylib.org/lilyglyphs                    %
%                                                                        %
%  Copyright 2012-2013 Urs Liska and others, ul@openlilylib.org          %
%                                                                        %
%  'lilyglyphs' is free software: you can redistribute it and/or modify  %
%  it under the terms of the LaTeX Project Public License, either        %
%  version 1.3 of this license or (at your option) any later version.    %
%  You may find the latest version of this license at                    %
%               http://www.latex-project.org/lppl.txt                    %
%  more information on                                                   %
%               http://latex-project.org/lppl/                           %
%  and version 1.3 or later is part of all distributions of LaTeX        %
%  version 2005/12/01 or later.                                          %
%                                                                        %
%  This work has the LPPL maintenance status 'maintained'.               %
%  The Current Maintainer of this work is Urs Liska (see above).         %
%                                                                        %
%  This work consists of the files listed in the file 'manifest.txt'     %
%  which can be found in the 'license' directory.                        %
%                                                                        %
%  This program is distributed in the hope that it will be useful,       %
%  but WITHOUT ANY WARRANTY; without even the implied warranty of        %
%  MERCHANTABILITY or FITNESS FOR A PARTICULAR PURPOSE.                  %
%                                                                        %
%%%%%%%%%%%%%%%%%%%%%%%%%%%%%%%%%%%%%%%%%%%%%%%%%%%%%%%%%%%%%%%%%%%%%%%%%%

\documentclass[oneside]{article}
\usepackage{lilyglyphsStyle}
\usepackage{lilyglyphsManualFonts}
\begin{document}
\begin{center}
{ \Huge \lilyglyphs }

\bigskip
{ \Large Urs Liska }

\emph{September 2013} 

\end{center}

\bigskip
\lilyglyphs{} makes the notational elements of LilyPond%
\footnote{\url{http://www.lilypond.org}}
available to \LaTeX{} documents.
This can be accidentals like \flat{} or \sharp, but also articulation scripts like \hspace{.1ex} \fermata, notes like \crotchet{} or time signatures like \lilyTimeCHalf{} or \lilyTimeSignature{5}{8}.
This will be an important tool if you author documents about music, be it a critical report or an analytical essay.
Now it's easy to write: “in m.\,24 the \decrescHairpin{} lasts from the 3rd to the \lilyDynamics{sf} on the 5th \semiquaverDown”.

One thing that sets \lilyglyphs{} apart from other solutions I had investigated is that it effectively can print \emph{anything} that can be done with LilyPond. 
Not only accesses it the OpenType music font provided by LilyPond, but it also allows one to use small pdf images that are generated by LilyPond, thus enabling the use of arbitrarily complicated constructs.
There is a (growing) set of predefined commands available while anything else can be used using generic access commands.

The other nice thing is that the glyphs scale well with the surrounding font size, making it easy to incorporate them into continuous text.
By default the glyphs scale automatically but they can also be scaled manually, individually or for the whole document.

\end{document}