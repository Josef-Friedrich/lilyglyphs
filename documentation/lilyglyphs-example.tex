%%%%%%%%%%%%%%%%%%%%%%%%%%%%%%%%%%%%%%%%%%%%%%%%%%%%%%%%%%%%%%%%%%%%%%%%%%
%                                                                        %
%      This file is part of the 'lilyglyphs' LaTeX package.              %
%                                ==========                              %
%                                                                        %
%              https://github.com/openlilylib/lilyglyphs                 %
%               http://www.openlilylib.org/lilyglyphs                    %
%                                                                        %
%  Copyright 2012-2013 Urs Liska and others, ul@openlilylib.org          %
%                                                                        %
%  'lilyglyphs' is free software: you can redistribute it and/or modify  %
%  it under the terms of the LaTeX Project Public License, either        %
%  version 1.3 of this license or (at your option) any later version.    %
%  You may find the latest version of this license at                    %
%               http://www.latex-project.org/lppl.txt                    %
%  more information on                                                   %
%               http://latex-project.org/lppl/                           %
%  and version 1.3 or later is part of all distributions of LaTeX        %
%  version 2005/12/01 or later.                                          %
%                                                                        %
%  This work has the LPPL maintenance status 'maintained'.               %
%  The Current Maintainer of this work is Urs Liska (see above).         %
%                                                                        %
%  This work consists of the files listed in the file 'manifest.txt'     %
%  which can be found in the 'license' directory.                        %
%                                                                        %
%  This program is distributed in the hope that it will be useful,       %
%  but WITHOUT ANY WARRANTY; without even the implied warranty of        %
%  MERCHANTABILITY or FITNESS FOR A PARTICULAR PURPOSE.                  %
%                                                                        %
%%%%%%%%%%%%%%%%%%%%%%%%%%%%%%%%%%%%%%%%%%%%%%%%%%%%%%%%%%%%%%%%%%%%%%%%%%

\documentclass[oneside,11pt]{article}
\usepackage{lilyglyphsStyle}
\usepackage{lilyglyphsManualFonts}

\pagestyle{empty}

\newcommand{\superscript}[1]{{\addfontfeatures{VerticalPosition=Superior}#1}}
\linespread{1.1}

% make the glyphs lighter
\lilyOpticalSize{26}

\begin{document}
\begin{center}
{ \Huge \lilyglyphs }

\bigskip
{ \Large Urs Liska }

\emph{September 2013} 

\end{center}

\bigskip

You are authoring text documents about music, maybe you're a musicologist, teacher, or composer?
You are preparing such documents for publication and have always missed the ability to print sentences like this one?

\begin{quote}
In m.\,24 the \decrescHairpin{} lasts from the 2\superscript{nd} \halfNote{} to the \lilyDynamics{sf} on the 11\superscript{th} \semiquaverDown[raise=-.5].
\end{quote}

With the new package \lilyglyphs{} you can easily insert the notational elements of the LilyPond%
\footnote{\url{http://www.lilypond.org}}
engraving software in your text documents.
Accidentals like \flat{} or \sharp, but also articulation scripts (\hspace{.65ex}\fermata{}) and time signatures such as \lilyTimeCHalf{} or \lilyTimeSignature{5\,+\,7}{8} are readily available.
But you can also insert arbitrary notational constructs like this mockup example \lilyFancyExample{} into your text or even make scanned images available as “characters”.
This package may greatly extend your typographical options when authoring or typesetting critical reports, analytical texts or teaching/exam material.

One thing that sets \lilyglyphs{} apart from other solutions I had investigated is that one isn't restricted to a set of predefined symbols but is able to print \emph{any notation} that can be realized with LilyPond. 

\footnotesize
The other nice thing is that the glyphs \halfNoteRest{} scale well with the surrounding font size, 
\large making it easy \halfNoteRest{} to incorporate them into continuous text.
\normalsize By default the glyphs scale automatically but they can also be scaled \clefF[scale=.4] manually, either individually or document-wise \clefG[scale=1.5,raise=-3.1].

\medskip
Sounds too good to be true? Well, there’s one catch to it: \lilyglyphs{} is a \LaTeX{}%
\footnote{\url{http://www.latex-project.org}}
package, so you may have to consider a change in mind-set. 
But if the above examples whet your appetite and match your professional needs you should really consider giving it a serious try -- it's all Free Software anyway.
You may find reading my essay on the advantages of plain text toolchains helpful too%
\footnote{\url{http://lilypondblog.org/2013/07/plain-text-files-in-music/}}.
 
For more information you may visit \lilyglyphs' project homepage \url{http://openlilylib.org/lilyglyphs} or contact us through \href{mailto:info@openlilylib.org}{info@openlilylib.org}.

\end{document}