\documentclass{OLLbook}
%\usepackage{microtype} 
\usepackage{lilyglyphsStyle}
\usepackage{hologo}  % to typeset XeLaTeX and LuaTeX logos regardless of engine used
\begin{document}

% Insert a picture with a glyph list from LilyPond's glyph list
% #1 is the caption of the figure
% #2 the label name
% #5 the file name of the image
\newcommand{\glyphlist}[3]{%
	\begin{figure}[htbp]
		\centering
		\includegraphics[width=.75\textwidth]{glyphlist/resources/#3}
		\caption{#1}
		\label{fig:#2}
	\end{figure}
}

%%%%%%%%%%%%%%%%%%%%%%%%%%%%%%%%%%%%%%%%%%%
% Title page
%%%%%%%%%%%%%%%%%%%%%%%%%%%%%%%%%%%%%%%%%%%

\setbookmaintitle{The \lilyglyphs[raise=0.1]Package\\~\\
	\normalsize Version 0.2.0}
\author{Urs Liska}

\maketitle

\pagebreak

\tableofcontents
\vfill

\pagebreak

\begin{authorAbstract}{Urs Liska}
\lilyglyphs is a package that allows to include notational elements from the exceptional LilyPond notation software as standard elements in the continuous text of \hologo{XeLaTeX} documents.

\lilyglyphs is part of the \texttt{openLilyLib} project%
\footnote{\url{www.openlilylib.org}}, 
a family of resources for the LilyPond notation software and \LaTeX{} authoring about music.
\end{authorAbstract}

\vfill
\input{copyright-notice.inp}

\chapter{Introduction}
\label{chap:introduction}
The package \lilyglyphs is made for authors who want to include single musical elements in their \LaTeX{} texts. 
This is not meant to be used for full-fledged musical examples with staves, for which one would use packages like \texttt{musixtex} or the LilyPond software together with lilypond-book\footnote{\url{http://www.lilypond.org}}. 
Instead we are talking about inserting items like \quaver or \lilyRFZ within the paragraphs of continuous text.

To achieve this \lilyglyphs aims at making the notation font and notational elements of the LilyPond notation software available to \LaTeX{} documents.
LilyPond is a notation software package that is a competitor in producing the most beautiful musical engraving on the market, and one of the foundations of this beauty is its \emmentaler font.
The \emmentaler fonts are accessed through the \texttt{fontspec} package, and \lilyglyphs therefore relies on a \LaTeX{} distribution supporting \texttt{fontspec}. 
It was written using \hologo{XeLaTeX}, but should also work with LuaLaTeX -- although this hasn't been tested%
\footnote{Principally it is possible to extend the functionality so it would also work with plain \LaTeX, but as the package maintainer has neither experience nor active interest in this area this will only be implemented if there are volunteers who join us.}%
.

The \emmentaler font only provides a subset of LilyPond's notational capabilities, as LilyPond creates many symbols by combining glyphs and drawing explicitely.
To also access this kind of symbols we have chosen an approach that uses small pdf images to print glyphs not present in \emmentaler*.
There is an accompanying Python script to make the creation of new arbitrary symbols as easy as possible.

The \lilyglyphs package was initiated and is maintained by Urs Liska, and is hosted as a git repository on GitHub\footnote{\url{http://www.github.com/lilyglyphs/lilyglyphs} -- Contact: \url{mailto:lilyglyphs@ursliska.de}}.
If you would like to participate in its development, don't hesitate to contact us.

\bigskip
For activating the capabilities of \lilyglyphs just write \cmd{usepackage\{lilyglyphs\}} in the preamble of your document. 
Then you can use the commands defined by the package or access glyphs directly through one of the generic access commands.
Both are described in detail later, in this section you'll only find a general overview of the possibilities.

Glyphs are scaled to fit normal text fonts, and the scaling automatically follows the scaling of the text. 
This also means they are influenced by \LaTeX 's text size commands, so you can write something like \{\cmd{Large \textbackslash flat\}} to create a larger {\Large \flat}glyph than the normal \flat one or even something really \cmd{Huge} {\Huge \lilyDynamics{sfz}} -- which will probably spoil your line spacing if used within a paragraph. 
You can influence the size and vertical position of the glyphs -- with the predefined commands as well as the generic ones -- individually or globally through the use of optional arguments.
This way you can also accomodate unusual text fonts that might otherwise not go well together with \lilyglyphs*.

You can always combine commands (or glyphs created through generic access as shown in the next section) to create new commands for later reuse. 
But it is better to follow the documentation for package designers to create such commands in the way suggested also for package designers in \fref{sec:howto_predefined_commands}.
If you feel you have specific commands that would be a valuable addition to the package please don't hesitate to contact us.

\section{Installation}
\label{section:installation}
As the procedures necessary to make this package work for you may vary depending on your operating system and \LaTeX{} distribution, we can only give you some general hints on how to set up your system.

First of all you have to decide how you obtain the package.
There are two main ways, and both have their pros and cons.
You can either clone the GitHub repository or download one of the archived versions of the package.

The recommended way to get \lilyglyphs is to clone the GitHub repository.
If you consider contributing to the package it is preferable to first fork the repository, then clone into the fork and add the original repository as an additional remote.
This way you can easily apply any changes, push them to your fork and send us a pull request.

There is one important issue to consider using the original repository: You will need working installations of both LilyPond and Python on your system to set the package up.
This is because the pdf files that are used for the graphics driven commands aren't tracked in Git.
When you have cloned the repository for the first time or have fetched updates that include new image files, you will have to run the \texttt{rebuild-pdfs.py} script in the \texttt{/py} scripts directory of the package.
This will run LilyPond multiple times to create the necessary image files.
Without these files the respective commands won't compile in your \LaTeX{} documents.
And if you think you might do without this subset of the available commands, consider that you won't even be able to compile the manual from its .tex source, which isn't included as a PDF file either -- another issue to consider.
But since you are reading this, you will probably have already managed to download the PDF version from GitHub \dots

The big advantage of directly using the repository is that it is the easiest way to get updates that might occur quite often. 
Each time someone adds new items to the list of commands they will be instantly accessible through the repository, but regular releases will happen much less frequently.
If updates should somehow break things in existing documents you can easily (and temporarily) check out the earlier version of the package that you used when originally writing your documents.
Another advantage of the repository version is that you have all the tools at hand (especially the Python scripts) that make your life easier if you want to extend the package four yourself.

If you know that you will only \emph{use} the \lilyglyphs package, then you can download one of the archived versions from the download page on the package's GitHub site.
These versions contain all necessary binary files to use the package from your documents, but strips off any supplementary files that you might need to extend the package.

An “intermediate” solution would be to download an automatically archived version of the complete repository. 
This will contain the scripts and the supplementary files, but not the binary files.
You should be able to download archives with the binary files for any released version from the download page.

\medskip
The initial requirement is to make the package file (lilyglyphs.sty) available to your \LaTeX{} distribution. \textit{The usual way to achieve this is to extract the archive or clone the repository to a place in the \LaTeX{} search path. This might for example be a subfolder of \texttt{texmf/tex/latex} in your home directory. On Linux you'd have to run \texttt{texhash ~/texmf} afterwards to update \LaTeX's cache.}

Of course the fonts have to be accessible to \texttt{fontspec}. 
Please refer to the \texttt{fontspec} documentation on how to make fonts visible to it. 

\textit{On a Linux system it may be enough to copy the \emmentaler font files to your home directory's .fonts directory or in a new subdirectory of it, possibly running \texttt{fc-cache} afterwards to update the fontconfig cache. 
The font files are located in the usr/share/lilypond/current/fonts/otf subdirectory of your LilyPond installation. 
Eight .otf files (representing different optical sizes, see \fref{sec:optical_size}) are also included in the distribution archive of \lilyglyphs}.

If you experience any issues during installation that would call for diferrent instructions then please provide us some feedback.

\chapter{Usage}
\label{chap:usage}

\section{Usage of predefined commands}
\label{sec:usage_predefined_commands}
In order to use predefined commands that have already been imlemented you generally just have to enter the command to print the corresponding musical element, e.\,g.\ \cmd{lilyTimeC} for a \lilyTimeC*.

The predefined commands are generally available in two forms, starred and unstarred (e.\,g.\ \cmd{flat} and \cmd{flat*}). 
The unstarred commands append a single space after the glyph while the starred ones provide the glyphs without trailing space. 
Use the unstarred versions usually in the continuous text, and the starred ones before punctuations, or if you want to combine multiple glyphs.
There are some commands that don't need this differentiation -- notably Dynamics letters and numbers -- because the underlying access technique works differently.
For these commands you can just put whitespace after the closing curly brackets or not and \LaTeX{} will respect this.
The documentation of the predefined commands always tells you whether there are starred versions or not.

\section{Generic access commands}
\label{sec:generic_access_commands}
To print \emmentaler glyphs that aren't covered by the predefined commands yet, \lilyglyphs offers four generic access commands:\\
\cmd{lilyGlyph}, \cmd{lilyText} \cmd{lilyGlyphByNumber} and \cmd{lilyImage}. 

Their mandatory argument is the content to be printed, which has to be given in a form specific to the respective command.

\medskip
\cmd{lilyGlyph} expects the OpenType glyph name. 
You can look up the glyph names in the Appendix of LilyPond's \emph{Notation Reference}\footnote{\url{http://www.lilypond.org/doc/v2.16/Documentation/notation/the-feta-font}} or the somewhat reduced html page provided in the package download. 
The current pdf manual for \lilyglyphs will eventually also contain a complete list of available glyphs, but this will only be updated every now and then (presumably when predefined commands of a new category are added).

\medskip
\cmd{lilyGlyphByNumber} expects the Unicode code of the glyph. 
You will generally not want to use this as the code positions aren't guaranteed to stay the same with new versions of the fonts. 
There may be some uses for numerical access however.

\medskip
\cmd{lilyText} expects ordinary text as its argument. 
In fact it just switches the font to \emmentaler and then writes the string given as the argument. 
This only works for Dynamics letters, numbers and the glyphs \lilyText[scale=2]{+ - , .} -- as these glyphs are located at their ordinary character position in \emmentaler*. 
But you can also enter any spacing commands (like \cmd{hspace} or plain spaces). 
Keep in mind that this may result in line breaking inside your expression. 
If you need to prevent this you can surround your expression by a \cmd{mbox}.

\medskip
\cmd{lilyImage} expects the basename of a usable image file.
It then includes this file using the same optional argument mechanism as the other commands.
What is somewhat special about this command is that it automatically scales the image relative to the current text font size, with being printed at its original size with \cmd{normalsize}.
You have to take care yourself that \LaTeX{} finds and can handle the image file.


\section{The optional argument: Scaling and placement}
\label{sec:optional_argument}
The generic access commands as well as the predefined commands allow an optional argument to be passed.
This can contain a list of comma-separated options in \texttt{<key=value>} form that influence the appearance of the glyphs. Currently there are the \texttt{scale} and \texttt{raise} options.

\medskip
\texttt{\textbf{scale}} changes the size of the glyph. 
As the \emmentaler glyphs are designed for a totally different purpose they often don't fit very well in the context of continuous text.
\texttt{scale} is given as a factor by which the default size is multiplied.
With \emmentaler glyphs this has to be a positive number, otherwise you will get an error.
But glyphs printed by \cmd{lilyImage} (or predefined commands based on it) can also be scaled negatively.
This results in an image that is rotated around the center of the bottom line of the original.
You will therefore have to add an appropriate \texttt{raise} value (try e.\,g. 2 as a starting point).
You will have to take some care about the horizontal spacing, as such a flipped image seems to use it with inverted direction.
But you can safely put extra horizontal space after the image, and it is a valid and practical way to create two symbols from one image file.


\medskip
\texttt{\textbf{raise}} changes the vertical placement of the glyph.
The majority of glyphs is placed too low, so they need a positive \texttt{raise} value.
\texttt{raise} is given as a decimal value without units, which is interpreted as \emph{ex}, or x-height. 
As there is no \emph{x} in a musical font, this is somewhat arbitrary, but it is a natural unit to scale with the font size.
Usually you may start trying \texttt{raise} values between 0 and 0.5.

\medskip
There are three levels where this influence may be effective: global setting, design time (with predefined commands) and command invocation.

At a \textbf{global level} the options are set to defaults of \texttt{scale=1} and \texttt{raise=0}, which basically means that the glyphs are unaltered.
You can change these globally effective options at any time with \cmd{lilyGlobalOptions\{<options>\}}.
This may for example be necessary document-wide if you use a text font which doesn't harmonize well with \lilyglyphs*' default settings.
But you can also use it if you for example need much bigger musical glyphs for presentation purposes.

At \textbf{design time} the package designers set the default options for their predefined commands, so they work right out of the box in most cases.

When \textbf{invoking} the commands within an actual document you can pass the options for the specific instance of the glyph.

The effective option values that are applied to a specific instance are calculated from all three levels. 
So if you pass an option during the command invocation, you don't set an absolute value, but modify the value already present. 
An option \texttt{scale=1.2} in a command doesn't mean that you apply 1.2 as the scaling factor to the original glyph, but that the glyph is printed 1.2 times the size it would have been printed without this option.
So if you want to slightly increase the size of a single glyph you may pass 1.1 as the value to \texttt{scale}, even if you have earlier set the global scaling to 1.8 -- the effective scaling will be \mbox{1.1 * 1.8.}

\textit{Technically speaking \lilyglyphs applies a fourth layer of scaling with image files.
It calculates a last scaling factor by multiplying the result of the above considerations with the ratio of the current font size versus the size of the \cmd{normalsize} font.}


\section{Examples}
\label{sec:examples}
Now it's time to see how you actually print \emmentaler glyphs using the generic access commands.

The \emph{fermata} sign isn't implemented yet as a predefined command, so you can print it with \cmd{lilyGlyph\{scripts.ufermata\}}: \lilyGlyph{scripts.ufermata} -- we have looked up the name of the glyph in the documentation.
 
As this doesn't well fit in the line of text you will want to adjust its size and placement.
This is done with the optional argument described in the previous subsection.

First we increase the size of the glyph with the \texttt{scale} argument. We find that a scaling factor of 1.4 seems suitable:\\
\cmd{lilyGlyph[scale=1.4]\{scripts.ufermata\}} -- \lilyGlyph[scale=1.4]{scripts.ufermata}

As you can see the glyph is -- as most \emmentaler glyphs are -- placed too low, so you have to add the \texttt{raise} argument. A value of 0.3 seems fine -- remember, the raise argument is interpreted as \emph{ex}, but you don't write down the unit.\\
 \cmd{lilyGlyph[scale=1.4,raise=0.3]\{scripts.ufermta\}} -- \lilyGlyph[scale=1.4,raise=0.6]{scripts.ufermata}
 
You can now further see that the glyph is placed too far to the left -- which is probably due to the fact that articulations are originally intended to be centered relative to their reference point (the note).
So you have to add some more space before, which might be rational to enter in \emph{ex}:\\
\cmd{hspace\{1ex\}}\cmd{lilyGlyph[scale=1.4,raise=0.3]\{scripts.ufermta\}} -- \hspace{1ex}\lilyGlyph[scale=1.4,raise=0.3]{scripts.ufermata}

\medskip
If you want you can now simply enclose this definition in a \cmd{newcommand}  to be able to reuse it.
However you are encouraged to follow our guidelines on how to create predefined commands as explained in \fref{sec:howto_predefined_commands}.
If you manage to write a command that you find useful for others also please submit it to us -- or even better: if you figured out how to create commands in general, please join us.

\section{Dotted symbols}
\label{sec:dotted_symbols}
With notes and rests you will find \cmd{...Dotted} versions of commands.
If you use significant scaling factors for the commands you have to take care about the gap between the glyph and the dot, because this sometimes doesn't scale too well.
Unfortunately one can influence neither the gap nor the scaling and raise factors of the dot independently.
We have worked hard to enable the designer of a command to create rules how to scale the gap, but you still may run into problems here. 
In such a case you will have to either change the predefined command in the library or just create the dotted symbol from scratch.

\cmd{...Dotted} commands have starred and unstarred versions.

You can write arbitrarily dotted symbols by adding \cmd{lilyPrintMoreDots} after the dotted symbol. 
This command uses the existing dot settings (scale and raise) and prints another dot. 
By default it has a gap of 0.25ex, but you can override this by passing a number as an optional argument, which is interpreted as \emph{ex}.
For example if you take the command \cmd{halfNoteRestDotted}, which prints a dotted half note rest: \halfNoteRestDotted  you can easily add more dots through \cmd{halfNoteRestDotted*\textbackslash lilyPrintMoreDots}: \halfNoteRestDotted*\lilyPrintMoreDots (note the use of the starred and unstarred versions).

Please note that you should only call \cmd{lilyPrintMoreDots} immediately after calling a \cmd{...Dotted} command. Otherwise you may get surprising results or even errors because the underlying key-value variables are inititialized wrongly or not at all.

\section{Optical size}
\label{sec:optical_size}
The \emmentaler fonts come in a set of eight “optical sizes”.
These are variations of the font originally designed to be used at different point sizes.
Generally you can assume that fonts for larger sizes are somewhat lighter, while fonts for smaller point sizes give more weight on the paper.

\lilyglyphs gives you the option to access the available font versions, but it may make more sense to appreciate them as “weights” -- although this is technically speaking or even conceptionally incorrect.
\emmentaler has eight versions: 11, 13, 14, 16, 18, 20, 23, 26. 
If you conceive these as weights you would somehow order them from black (11) to light (26).
You can switch the used optical size at any time in a document using the command \cmd{lilyOpticalSize}, giving the number as an option. 
You have to make sure to supply a number corresponding to a font actually available on your system.
Maybe this will someday also be available as an option to select for a single glyph.
For now you would have to switch twice, before and after the glyph.

The optical size used by \lilyglyphs defaults to 16.

\paragraph*{Known issues and warnings:}
Optical sizes don't work with glyphs printed as images.
If you \emph{have} to use these glyphs in different weights, you will have to take care for it yourself.
The general plan would be to create different versions of the glyph by creating different glyphs in LilyPond (presumably by using different staff sizes).

\chapter{Internals}
\label{chap:internals}

\section{Documentation of the generic access commands}
\label{sec:internals_generic_access}
This section is essential for readers who want to understand how this package works internally, for example if they want to actively participate in its development.
And it is recommended (but not exactly mandatory) for readers who want to write their own predefined commands or to contribute to the package by adding commands to it. 
The latter target group might be satisfied reading the next section about writing predefined commands, but understanding will be easier and deeper with this subsection.

\medskip
In order to make the package .sty file easier to understand, its contents are split into multiple input files which are located in the \texttt{/commands} and \texttt{/core} subfolders.
The most fundamental definitions are in the \texttt{keyval.inp} and \texttt{genericAccess.inp} files.

\subsection{Accessing \emmentaler Glyphs}
\label{subsec:accessing_emmentaler_glyphs}
The command that actually prints glyphs from the \emmentaler fonts is \cmd{lilyPrint}, defined in \texttt{genericAccess.inp}.
It isn't intended to be called directly within a document, but only from the predefined commands.
It takes two arguments, the first -- optional -- being the comma-separated list of \texttt{<key=value>} pairs, the second the actual content to be printed.

\begin{verbatim}
\newcommand*{\lilyPrint}[2][]{%
    \interpretLilyOptions{#1}%
    \raisebox{{\lilyEffectiveRaise}ex}{%
        {\fontspec[Scale=\lilyEffectiveScale]{Emmentaler-\lilyOpticalSuffix}#2}%
    }%
}
\end{verbatim}

At first the command \cmd{interpretLilyOptions} is called, where the options of the different levels are evaluated and calculated to their effective values.
Then the content of \#2 is printed, within a \textbackslash raisebox and with the currently selected opticals version of the \emmentaler font.

\begin{verbatim}
\newcommand*{\interpretLilyOptions}[1]{%
    \setkeys{lilyCmdOptions}{scale=1,raise=0}%
    \setkeys{lilyCmdOptions}{#1}%
    \pgfmathsetmacro{\lilyEffectiveScale}{%
        \lilyGlobalOptions@scale * \lilyCmdOptions@scale * \lilyDesignOptions@scale}%
    \pgfmathsetmacro{\lilyEffectiveRaise}{%
        \lilyGlobalOptions@raise + \lilyCmdOptions@raise + \lilyDesignOptions@raise}%
}
\end{verbatim}
\cmd{interpretLilyOptions} is defined in \texttt{keyval.inp}.\\
The \texttt{<key=value>} mechanism is achieved using the \texttt{keyval} package as the most basic solution available.
{\color{red} If this can be implemented more elegant, extensible or powerful using other packages, e.\,g.\ \texttt{pgfkeys}, we'd appreciate any input.}
It uses three families of keys, corresponding to the three levels of options:
\texttt{lilyGlobalOptions}, \texttt{lilyDesignOptions} and \texttt{lilyCmdOptions}.

In a first step the keys for the actual command options are initialized to a neutral state.
This is necessary because otherwise options that aren't actually present in the command invocation were in an uninitialized state or in the state set by the last occurence of the option.
After this the command options are set to the actual state given in the command (i.\,e. the command the end user writes in the document).
Options that are explicitely given override the default state while options that are not present don't affect it.
Finally the effective values of the options are calculated from the global, the design and the command options. 
The scaling values are multiplied, the raise values added.
While the command options have just been determined, the global options are valid globally (and can be changed globally) and the design options have been set by the command that actually called \cmd{lilyPrint}.
This is the reason why \cmd{lilyPrint} should never be invoked directly -- the design options would be in the unknown state of the previous invocation of\cmd{lilyPrint}.

\medskip
The next higher level are the three generic access functions \cmd{lilyGlyph}, \cmd{lilyGlyphByNumber} and \cmd{lilyText}, defined in \texttt{genericAccess.inp}.
They are very similar and differ only in the way they determine the actual content to be printed.
As stated in the end user part of this documentation they expect two arguments, the optional \texttt{<key=value>} pair list and the contents.
As a first step the commands initialize the design options to a neutral state, because the “design” of the generic glyphs has to be neutral by design.
In the second step they invoke \cmd{lilyPrint}, passing the optional argument along and determine the printed content individually:
\cmd{lilyGlyph} calls the helper function \cmd{lilyGetGlyph}, \cmd{lilyGlyphByNumber} calls \cmd{lilyGetGlyphByNumber}, while \cmd{lilyText} just passes its contents argument unchanged to \cmd{lilyPrint}.

\medskip
These helper functions are important because most predefined commands call one of them to select glyphs from the \emmentaler fonts.\\
\cmd{lilyGetGlyph} takes the glyph name as found in the LilyPond documentation.\\
\cmd{lilyGetGlyphByNumber} takes the Unicode character index of the intended glyph.
But be aware that the Unicode index may change at any time with new versions of the \emmentaler font, so it usually isn't a good idea to access glyphs through their index.
There may be some uses for numerical access, however, e.\,g.\ to iterate through a range of glyphs.

\subsection{Printing image files}
\label{subsec:printing_image_files}
\cmd{lilyPrintImage}, defined in \texttt{genericAccess.inp}, is the command that prints glyphs from a supplied image file.
It actually is quite similar to \cmd{lilyPrint}, with only the extra consideration of scaling the image to the text font size.

\begin{verbatim}
\newcommand*{\lilyPrintImage}[2][]{%
    % interpret optional argument
    \interpretLilyOptions{#1}%
    % determine scaling factor to accomodate the current font size 
    % (as images don't scale automatically with the font)
    \lilyScaleImage%
    % Print the image in a raisebox
    \raisebox{{\lilyEffectiveRaise}ex}{%
        \includegraphics[scale=\lilyImageEffectiveScale]{#2}%
    }%
}
\end{verbatim}
First the command calls \cmd{interpretLilyOptions}, which is the same as with \cmd{lilyPrint}.
But as an additional step \cmd{lilyScaleImage} is called, and finally it uses \cmd{lilyImageEffectiveScale} as the scaling factor instead of \cmd{lilyEffectiveScale}.
I won't explain this in detail, but in effect it multiplies the \cmd{lilyEffectiveScale} calculated before with the ratio of the current font size to the \cmd{normalsize} size.

What is finally given as the content to be printed is the basename of an image file.
With \hologo{XeLaTeX} this can be a file in .pdf, .png or .jpg format, but we highly recommend using .pdf files for sake of printing quality.

As with printing \emmentaler glyphs there is no handling of design time options here, and for that reason you should never call this command directly from a document.
Please use \cmd{lilyImage} instead which does the same as \cmd{lilyPrintImage} but additionally defaults the design time options to neutral values.

\section{How to write predefined commands}
\label{sec:howto_predefined_commands}
Writing your own predefined commands is actually quite straightforward -- and identical if you want to write a command for your document or for inclusion in the package.
So if you find yourself creating predefined commands that you think are useful for general use, don't hesitate to submit them to us. 

\subsubsection{Commands that print single glyphs}
\label{subsubsec:commands_single-glyphs}
Let's review an example of a predefined command, the \cmd{doublesharp}.

\begin{verbatim}
% "accidentals.doublesharp"
\newcommand*{\doublesharpBase}[1][]{%
    \setkeys{lilyDesignOptions}{scale=1.5,raise=0.35}%
    \lilyPrint[#1]{\lilyGetGlyph{accidentals.doublesharp}}%
}

\newcommand*{\doublesharp}[1][]{\doublesharpBase[#1] }

\WithSuffix\newcommand\doublesharp*[1][]{\doublesharpBase[#1]}
\end{verbatim}
As you can see there are actually three commands, a base command and the starred and the unstarred versions.
But let's start with the original command.

We use the starred version of \cmd{newcommand}, because we always know that our commands are restricted to single paragraphs.
We declare that our command accepts one optional argument, which defaults to empty. 
This argument can take the list of \texttt{<key=value>} options.
When writing the commands, please take care not to omit the \texttt{\%} characters at the line endings, as they prevent unwanted whitespace to be introduced in the output.

In the second line we define the design options for your command.
In the example the designer has decided that a doublesharp glyph should be scaled to 1.5 and placed 0.35 ex above its default level.

The third line calls the internal \cmd{lilyPrint} command. 
It passes the optional argument, with which the end user can override (i.\,e.\ modify) the designed values.
As the \doublesharp is a glyph that has to be selected by its glyph name, we call \cmd{lilyGetGlyph}, supplying the glyph name found in the documentation.
The result of this command is passed as the \texttt{\#2} to \cmd{lilyPrint}.

To summarize: Writing a predefined command for printing a glyph from \emmentaler involves just two steps, setting the design time options and calling \cmd{lilyPrint} with the appropriate \#2 argument.

\medskip
If we have done it right, the new command prints the desired glyph without any trailing space. 
The side-effect is that \LaTeX{} will ignore any whitespace after the command.
While it is always possible to write a pair of curly braces after or around the command to allow trailing whitespace, we decided to offer the starred and unstarred versions of commands.
The unstarred versions provide a trailing space and are therefore intended for general use in continuous text. 
The starred versions don't have this space and are intended for use before punctuation marks, in words or in combinations with other glyphs.
For these we need the \cmd{WithSuffix} construct that is shown in the example.
We name the commands identical to the original command except for the trailing *.
We give the same set of one optional argument, and in the command itself we just call the base one, passing the \texttt{\#1} argument along and add a trailing space.

\medskip
If you know the Unicode number of the desired glyph you can call \cmd{lilyGetGlyphByNumber} instead of \cmd{lilyGetGlyph}, but you can't be sure this number will stay the same forever.

\medskip
Creating commands using image files is practically the sameand even simpler.
If you look at the definition of a \cmd{crotchet}, 
\begin{verbatim}
\newcommand*{\crotchetBase}[1][]{%
    \setkeys{lilyDesignOptions}{scale=0.9,raise=-0.2}%
    \lilyPrintImage[#1]{crotchet}%
}
\newcommand*{\crotchet}[1][]{\crotchetBase[#1] }
\WithSuffix\newcommand\crotchet*[1][]{\crotchetBase[#1]}
\end{verbatim}
you will notice that the only differences are that the actual printing is done with \cmd{lilyPrintImage} instead of \cmd{lilyPrint} and that therefore the basename of the image file can be passed directly.
In \fref{chap:buildglyphimages} you will see a tool that allows to create numerous image commands quite easily.

\bigskip
As a last example we will look at the definition of \lilyRFZ \cmd{lilyRFZ}.

\begin{verbatim}
\newcommand{\lilyRFZ}[1][]{%
    \mbox{%
        \lilyDynamics[#1]{r\hspace{0.035ex}fz}%
    }%
}
\end{verbatim}
As mentioned in \fref{sec:dynamic_text}, \cmd{lilyDynamics} is just a wrapper around \cmd{lilyText}, setting the \texttt{scale} factor to 1.5.
While the other generic commands only print single glyphs, \cmd{lilyText} can print 'plain text', so usually there is no need to write predefined commands only to combine letters to a single command.
In some cases this may however be necessary.
In the given example of \cmd{lilyRFZ} we need to apply a little bit of extra space between the \lilyDynamics{r} and the \lilyDynamics{z}.
We see that we can insert a \cmd{hspace} command between the letters without any problems.
But as it turns out \LaTeX{} may now decide to insert a line break at any time, so we have to additionally enclose this call to \cmd{lilyDynamics} in a \cmd{mbox}.
The command just passes the optional argument to \cmd{lilyDynamics}, so you can use these arguments as usual.
This example is meant to encourage you to experiment with the definition of predefined commands. This is pretty much the same as usual when inventing new commands. You just have to deal with the design time options, the optional argument and getting the correct input as the second argument.

If you had created this command you would have noticed that you can't get any whitespace after it -- it is silently ignored.
So you would want to create starred and unstarred versions of the command, which works as in the first example. 
In \lilyglyphs this is in fact implemented this way.


%%%%%%%%%%%%%%%%%%%%%%%%%%%%%%%%%%%%%%%%%%%%%%%%%%%%%%%%%%5
% Appendix: Licenses

\subsubsection{Create Dotted Symbols}
\label{subsubsec:dotted_symbols}

It is not exactly trivial to  create dotted symbols as predefined commands.
Of course you can always use \cmd{lilyDot} to print LilyPond's dot glyph, but if you want to create commands that combine a glyph and one ore more dots you encounter two difficulties:
You can't apply the optional arguments independently on the two items, and there may be issues with the scaling and the gap between the two glyphs.

There is some infrastructure in \lilyglyphs (defined in the file dotted.inp) to faciliate dealing with dotted symbols, but this implementation may not be completely satisfactory so far.
Great care has been taken to hide as much complexity as possible in the mentioned file, in order to make the definition of actual commands as clean and concise as possible.

Let's analyze the implementation of a dotted half note rest:
\begin{verbatim}
% Dotted half note rest
\newcommand*{\halfNoteRestDottedBase}[1][]{%
    % define the optional arguments for the dot
    \setkeys{lilyDesignOptions}{scale=0.8,raise=0.2}%
    % Calculate effective scale/raise and the hspace for the dot
    \lilySetDotOptions[#1]{0.05}{0.5}{0}%
    % Print the rest and then the dot
    \halfNoteRest*[#1]\lilyDotSpace\lilyPrintDot
}
\end{verbatim}

As you will note the command is a \texttt{...Base} command that will be have to be complemented by the starred and unstarred versions, but we will silently ignore this issue here (see \fref{sec:commands_emmentaler-glyphs}).

The command takes the usual optional argument that can contain \texttt{<key=value>} pairs.
This applies to the dotted symbol as a whole.
Please note that in the last line the predefined command \cmd{halfNoteRest[\#1]} is called and passed the optional argument.
You can only use this technique of creating dotted symbols on top of correctly implemented predefined commands.

The first thing you have to do is to define the DesignTimeOptions for the dot. They are relative to the original design of \cmd{lilyDot}, and you have to adjust them so the dot suits the main glyph in its size and vertical position (you can ignore the horizontal spacing for now):
\begin{verbatim}
    % define the optional arguments for the dot
    \setkeys{lilyDesignOptions}{scale=0.8,raise=0.2}%
\end{verbatim}
You should always set these options because otherwise you might get strange results or error messages.

The next step is to call a quite complex command \cmd{lilySetDotOptions} that sets several options for the dot:
\begin{verbatim}
    % Calculate effective scale/raise and the hspace for the dot
    \lilySetDotOptions[#1]{0}{0.5}{0.4}%
\end{verbatim}
This command takes one optional and three mandatory arguments.
The optional argument is just the one that is written in the \LaTeX{} document and that is passed into the function. The remaining three arguments control the horizontal spacing between the main glyph and the dot.
As we now have two individual elements we have to control the gap between them explicitely as it doesn't scale relative to the \texttt{scale} argument by itself.
The relation between the \texttt{scale} factor and the horizontal gap can be understood as a curve (mathematically spoken: a 2nd order function).

The first (mandatory) argument sets the intensity of the curve.
A value of 0 will result in a linear relation (no curve at all), that is when doubling \texttt{scale} the gap will be exactly twice as wide.
Positive values will result in larger gaps for larger \textbf{scale}s.
As this is a quadratic function you will want to start with very small values or 0.

The second argument sets the general (linear) steepness of the curve.
A value of 1 means that by increasing \texttt{scale} by 1 the gap will be wider by 1\emph{ex}.
0.5 seems a good starting point for this argument.

The last argument is an offset in \emph{ex} for the whole curve which is independent from the scaling.
You can use it to accomodate specifically wide or narrow glyphs.

\cmd{lilySetDotOptions} takes all these informations, calculates some settings for the dot an stores them in internal variables that can be used by subsequent commands.
Please understand that they may be partially or totally overwritten by the next use of \emph{any} predefined command.
So you have to call this command immediately before actually printing the dot, otherwise it may or may not provide satisfying results.

The final line actually calls three commands:\\
\cmd{halfNoteRest*[\#1]} prints the already defined main glyph.
Note the use of the starred version, because the trailing space of the unstarred version would make it much harder to determine the parameters for the gap.\\
\cmd{lilyDotSpace} prints a horizontal space that is determined by the previous call to \cmd{lilySetDotOptions}.\\
\cmd{lilyPrintDot} finally prints the dot with the settings just defined.

\begin{verbatim}
    % Print the rest and then the dot
    \halfNoteRest*[#1]\lilyDotSpace\lilyPrintDot
\end{verbatim}

This was an explanation from the perspective of designing new predefined commands.
If you want to know how this is implemented internally, please look at the generously commented file \texttt{core/dotted.inp}.

\paragraph*{Known issues and warning:} 
One issue that hasn't been addressed yet is the vertical placement of the dot when scaled.
The dot is positioned relatively to the baseline of the text, and the main glyph may have a different center point. 
So when scaling the main glyph may seem to behave differently from the dot.\\~\\
If you want to create a dotted version of a glyph that is printed from an image file it will generally be easier and more reliable to create a command using a new image file.


\subsubsection{Create Multiply Dotted Symbols}
\label{subsubsec:multiply_dotted_symbols}

There is no need to define additional versions of glyphs with more than one dot.
For this purpose we have implemented the command \cmd{lilyPrintMoreDots}.
This prints a dot with the same characteristics as the preceding one from a dotted command (but remember that there shouldn't be any calls to other predefined commands in betwen).
The horizontal gap between the dots scales linearly with a default of 0.25\emph{ex} per unit of \texttt{scale}.
But if you pass an number as an optional argument this is interpreted as a different gap in \emph{ex}.
This command is available in a starred and an unstarred version.
Use the unstarred version when you need a space after the symbol, the starred one when you have a punctuation or another dot after it.

\noindent \cmd{halfNoteRestDotted*}\cmd{lilyPrintMoreDots*}! 
\halfNoteRestDotted*\lilyPrintMoreDots*!\\
\cmd{halfNoteRestDotted*[scale=1.5]}\cmd{lilyPrintMoreDots} and 
\halfNoteRestDotted*[scale=1.5]\lilyPrintMoreDots and 


\chapter{Easily create commands with Python scripts}
\label{chap:python_scripts}
As we have seen in \fref{subsubsec:commands_single-glyphs} it isn't really complicated to create predefined commands.
But we have developed a set of tools and templates to streamline this process even further.
With them it is mostly a matter of telling what you want, naming it and running a Python script.
But keep in mind that you can always do it manually if the pre-set workflow imposes any restrictions that you don't want to accept.
And it is nevertheless a good idea to have read \fref{subsec:howto_predefined_commands} to have a good understanding of what is going on.

The process basically consists of writing a definitions file (something like a template), calling a script and then putting the results to a useful place.
However it slightly differs if you create commands for using images or for using \emmentaler glyphs.
Both ways require a working Python 2 installation to work, creating commands with images additionally involves running LilyPond.
And the necessary scripts (and partially source files) are only available in the Git repository version of the package.
If you don't have Git installed but want to profit from them you can download zipped versions of the complete repository from GitHub.
You can't create image glyph commands at all without having LilyPond installed on your system, but you don't need Python in order to create new commands -- it just provides a much more convenient workflow.

\section{Generating Commands with Image Glyphs}
\label{sec:generating_commands_image}

The heart of the toolset is the Python script \texttt{buildglyphimages.py}, which is located in the \texttt{/py} directory.
It reads in an input file with one or more command definitions in it, generates compilable LilyPond source files from them and uses LilyPond to create small pdf files containing the images to be included in the \LaTeX{} file.
Then the script creates a new \LaTeX{} file with command definitions and example text, which can then be used to fine-tune the commands.
Finally you will have to move the generated commands to some useful location.

But let us start with looking at the form of the input files.


\subsection{Preparing the Input Files}
\label{subsec:BGI_preparing_input}
\texttt{buildglyphimages.py} doesn't expect regular LilyPond source files as its input, but rather a file with one or multiple 'snippets' in it. 
We will later see that it makes sense to provide compilable Lilypond files, though.

Input definition files are to be stored in the subdirectory \texttt{definitions} of our \texttt{/glyphimages} directory.

The script parses the input file line by line until it finds a line that contains the special key \verb|%%lilyglyphs|.																																																					
Then it starts reading in an entry for a single glyph command.

Any subsequent line that starts with the percent sign as a LilyPond comment is interpreted as a comment.
This comment will later be used for marking the command in the \LaTeX{} file.
If a line contains the special key \verb|%%protected|, the script skips this command.
You should use this key whenever you consider a command finished.
This won't only save the time of the LilyPond compilation, but will also prevent the existing command to be generated again in your output file.

If a line contains one of the keys \verb|scale=| or \verb|raise=| the value after the equals sign is interpreted as a new default value that the generated command gets as its design time options.
These values are also kept for subsequent command entries until the file is finished or the script finds a new entry -- which would replace the value.
The idea behind this option is that it is likely that one defines a set of related commands within an input file.
And it is likely that these related functions share common default values.
For example all single notes were created with a default scaling of 0.75.
Without this option you would have to adapt your result file for any single command, with it you can decide which default values you prefer for the commands at hand. 
The setting of option defaults also works if a command is marked as “protected”.
Generally this is an advantage since you don't have to take care to correctly update the file when you mark something as protected.
But there may be cases where you will want to remove the defaults setting from a \texttt{\%\%protected} clause.

Afterwards you provide the actual LilyPond command in the form of a variable holding a musical expression.
The first line should contain exactly the name, the '=' and the opening curly brace.
The following lines are read as LilyPond code until the parser finds a line starting with the closing curly brace.

Here you can see a concise example of a lilyglyphs entry section:

\begin{verbatim}
%%lilyglyphs
% crotchet with upward stem
crotchet = {
  g'4
}
\end{verbatim}
The name of the variable is very important, because it will also be used as the file name of the image file and the \LaTeX{} command name.

\paragraph*{Warning:}
So far the parser isn't very smart.
It just assumes that the input file it parses is correct.
If you provide code with deviations from this explanation, the script will probably produce erroneous results or just stop working.
Probably it won't do any harm, but we can't make any promises on that.
Please consider the script as being in a \textbf{very experimental} state.\\


You can define as many entries in one file as you like -- they will all be processed at once. 
Of course it is recommended to combine a coherent set of commands.

Everything that is outside of the lilyglyph entries is ignored by the script, so you can make use of these places to make a usable LilyPond file out of it.
You can start off with the file \texttt{definitions/\_template.ly}.
After the definition of the lilyglyph entry it places a markup, then it copies the variable to a new variable named \texttt{symbol} and finally includes a score from a special include file \texttt{score.ily}.
This will print your variable as the only element in a new score block that is prepared not to print staff, time signature and clef.
You can repeat this several times and have LilyPond produce a sheet with all defined symbols.
This way you can finish your design in “pure LilyPond” before actually going on to the \LaTeX{} part.


\subsection{Running the Script}
\label{subsec:BGI_running}

If you have prepared your input file you can run the \texttt{/py/buildglyphimages.py} script, passing the file name (without the path) along with the option \verb|-i| (or \verb|--input|). The script will look for the filein the \texttt{/glyphimages/definitions} directory.

In a first step the program parses this file and extracts information about the command definitions that haven't been marked as protected. Then it generates one compilable LilyPond source file for each command and saves it to a subdirectory of \texttt{/glyphimages/generated\_src} .
Each file is named according to the name of the command you specified in the input definition file, and the subdirectory is named according to the input file.
So if you combined several glyphs in a file \texttt{singlenotes.ly} the results will be stored in a \texttt{singlenotes} subdirectory.
And if you run the script again on a file with the same name but other definitions in it, it will add the new items to the directory.

Now these generated source files are compiled using LilyPond. 
The command line option \verb|-dpreview| takes care of producing a pdf file with the smallest possible bounding box around its content.
The directory is then cleaned up, and the resulting small PDF files are moved to the corresponding subdirectory below \texttt{glyphimages/pdfs}.

The final task of the script is to generate a \LaTeX{} file that is stored as \texttt{/stash\_new\_commands/imgages/newImageGlyphs.tex}.
Any existing file (i.\,e. the result of the previous run) will be overwritten.


\subsection{Utilizing the results}
\label{subsec:BGI_utilizing_results}

If you have successfully run the script you have the following results:
\begin{itemize}
\item LilyPond source files in the \texttt{/glyphimages/generated\_src/INPUT\_FILE\_NAME} folder (one for each command),
\item PDF files in the \texttt{/glyphimages/pdfs/INPUT\_FILE\_NAME} folder and
\item the file \texttt{/stash\_new\_commands/images/newImageGlyphs.tex}.
\end{itemize}
You shouldn't touch the first two items but go on and open the .tex file.
Keep in mind that this file will be overwritten by the next run of the script, so if you want to make any changes and/or keep the generated contents, you should immediately rename the file.
It is a good idea to name it according to the definitions input file because these two files form a natural pair.
This file is a working \LaTeX{} document that uses the \lilyglyphs package.
It contains \LaTeX{} \cmd{newcommand} definitions with starred and unstarred versions, one for each definition of the input file that hasn't been skipped in the process.
The visible part of the document contains a documentation table (as used throughout this manual) with all generated commands and example text for each generated command.

So what are we going to do with this file?
Well, it depends on how/where you want to eventually use it, but the first step will be to fine-tune the commands.

The script generates default values for the design time values of the optional argument, and it would of course be purely random if it would be perfect right away.
The first thing the generated example text does is showing that the command actually works.
But its more important purpose is to print the new glyph in different contexts: in continuous text, before punctuations, at the beginning of a line etc.
You should use these blocks of example text to adjust the arguments in the \verb|\setkeys{lilyDesignOptions}| clause.
You should specifically keep an eye on spacing issues: Does the glyph affect line spacing, is the “kerning” of the glyph correct?
On the other hand you should probably try to keep corresponding glyphs at an equal scaling.
Besides tweaking the optional argument values you can add space before or after the glyph, possibly for the starred and unstarred commands independently.
Keep in mind that you may as well use negative \cmd{hspace}.

Sometimes you will find that a rather tall glyph is difficult to accomodate because you have to scale it down so much in order not to affect the line spacing that it looks unnaturally small.
In such a case you might consider going back to the LilyPond side to review the design of your glyph (for example the glyphs for the single note commands have shortened stems).
Please make sure that you understand the implications of the following section about partial processing.

\medskip
If you are satisfied with the new commands you will have to move them to a useful place.
In any case you should keep copies of the input files as well as the current .tex file as a reference.
If you have put the command definitions in another file (e.\,g. in one of the .inp files of the \lilyglyphs package) you have to remove them from the generated file, otherwise you will get compilation errors.

If you are just creating the commands for your personal use, you can either copy the command definitions to the preamble of your working \LaTeX{} document or to a style file you might maintain.
But of course we would be happy if you decided to contribute them to the package, for which there are several possible ways:

Probably the simplest way is to send the definitions file and your processed results file as an email to the package maintainer. He would then run the script to generate the intermediate LilyPond and the pdf files and copy your edited command definitions to appropriate places in the package.

If you are working on a fork of the GitHub repository (as is recommended for potential participators) you could do even better by incorporating the new commands directly and sending us a pull request.
This works by completing the following steps:
\begin{itemize}
\item Move/copy the command definitions (with all comments) to appropriate .inp files in the \texttt{commands} subdirectory of the package root.
If you find you should create a new .inp file because your commands belong to a new category please take the existing files as a model and add an appropriate \cmd{input} statement to \texttt{lilyglyphs.sty}.
Please leave your definitions file in the \texttt{glyphimages/definitins} folder, with all commands marked as 'protected' (see next section for details).
Please also leave the generated LilyPond files in their subdirectory (you would have a hard time finding them anyway).
For any new files you might create please add the usual copyright comment at the beginning.
\item \emph{Please} add documentation for your added commands in the manual (\texttt{lilyglyphs.tex}).
Find the appropriate subsection in the “Predefined commands” section and add information to it.
At least we need the entries in the documentation table, but if there is anything special to note about the commands please explain this too.
If you add to an existing subsection you may copy the rows from the table in the file generated by the script, otherwise you should copy the whole table (but update the caption and label fields appropriately).
\item If you are happy with the result you can push to your fork of the repository and send a pull request through the GitHub web site.
It would be nice if we would receive such a pull request as a single commit (use \texttt{rebase -i}), as this is much easier to check.
\end{itemize}


\subsection{Partial processing of the input file}
\label{subsec:BGI_partial_processing}

You are encouraged to put a coherent set of multiple command definitions in one definitions file and keep this file, as it is the source from that everything can be rebuilt at any time.
But there will be occasions when you don't want the Python script to do all the work over and over again.
If you add a new command to the input file (and you \emph{should} add it to an existing file if it belongs to the same category) you only want the new command to be processed.
Or if you are working on the fine-tuning of the commands (as described in the previous section) and decide that you have to change the LilyPond definition of a single glyph you also only want to reprocess this one.
For this purpose you can mark entries in the input definitions file as “protected”, in order to prevent them to be newly processed by the script.
To do this you enter a line containing \verb|%%protected| in your entry definition (note the double percent sign and the missing space after them).
You can see an example for this in the \texttt{\_template.ly} file -- because the example entry shouldn't be processed at all.
From now on the marked command won't be processed anymore.
This means that the LilyPond source file won't be generated again, LilyPond isn't run again for the script (which would be the most annoying thing), and the \LaTeX{} commands won't be generated again.
The output file will of course be overwritten (you have renamed it before re-running the script, isn't it?) but it won't be cluttered with commands that you have already dealt with earlier.


\subsection{Recreating Image Files}
\label{subsec:BGI_recreating_images}

There are occasions when you have to (re)create the pdf files that serve as the glyph images without having to regenerate the \LaTeX{} commands or the LilyPond source files.
If you clone the Git repository then the pdf files are not present because they aren't tracked by Git.
Or if you pull updates from the remote repository that introduce new glyph images the pdf files won't be included for the same reason.
In that case you can run the script \texttt{rebuild-pdfs.py}, which is also located in the \texttt{/py} directory.
This script essentially compares the directories with the generated LilyPond sources and with the created pdf files, and if it finds a source without a corresponding pdf it will call LilyPond to recreate it.
Of course this script will only work with a working LilyPond installation.



\section{Generating Commands for \emmentaler Glyphs}
\label{sec:generating_commands_emmentaler}

Generating commands that print \emmentaler glyphs is very similar to the process with images described in the previous section, and it actually is even simpler, because there are no intermediate LilyPond files and compilations involved.
All there is to it is: create an input definitions file, run the \texttt{/py/genGlyphCommands.py} on it and fine-tune the generated \LaTeX{} file.
So in this chapter we mainly discuss the differences to the process as was already described.


\subsection{Preparing the input file}
\label{subsec:GGC_preparing_input_file}

The structure of the input file for the \texttt{genGlyphCommands} script is actually simpler than that for the \texttt{buildglyphimages} script.
This is due to the fact that we wanted the latter to be valid LilyPond source file as well -- a restriction we don't need to consider here.
As there is no creation of secondary files with LilyPond involved, the input file is much less important than with the image glyphs process.
There you need to keep the input files in order to be able to re-create or change the image files at any given moment, but here you only need to keep them if you want to keep track of what you have done.

The script will look for the input file in the \texttt{/stash\_new\_commands/emmentaler} directory, so you have to save it to that directory.
It can have any name and any or no file extension.

The input file should consist of one ore more command definitions.
Each command definition is composed of a set of lines with \texttt{key=value} pairs and finished by an empty line.
Please note that you shouldn't use whitespace around the equals sign.
Python and \LaTeX{} style commands are ignored, so you can use them if you want.
But keep in mind that empty lines have the special meaning of closing a command definition.
Therefore it is important that your file ends with a complete empty line, otherwise the last entry will be discarded.

\medskip
A command entry consists of several mandatory or optional lines.
The order doesn't matter, but it is surely advisable to stick to one style.
The following items are possible/necessary:
\begin{itemize}
	\item \textsf{\textbf{cmd} (mandatory)}: Specifies the command name. 
	You have to make sure that it is a valid \LaTeX{} name and that it isn't in use already.
	\item \textsf{\textbf{comment} (optional)}: You can pass a single line comment that will be used before the command definition.
	(If you want to have a multiline comment instead you can insert line breaks with \cmd{n}.)
	\item \textsf{\textbf{element} (mandatory)}: The actual element to be passed to the internal printing functions.
	The possible kind of values depends on the type of the command, as described with these types.
	\item \textsf{\textbf{type} (mandatory, but defaulted)}: The type determines the internal printing command to be used with the command.
	Please refer to \fref{sec:generic_access_commands} for more information.
	The option is mandatory but defaults to “glyphname”, so you can skip it for the most common case that the glyph is called by name.

	\begin{itemize}
		\item \textbf{“glyphname”:} The glyph is selected by its glyphname, which is what you have to specify as “element” (e.\,g. “accidentals.sharp“).
		You can look the glyph names up in LilyPond's documentation or in the glyph list contained in the package.
		\item \textbf{“number”:} The Unicode number is used to determine the glyph.
		\item \textbf{“text”:} The content is passed as plain text (works only for Dynamic letters, numbers and + - , .)
		\item \textbf{“dynamics”:} The content is also passed as plain text, but the scaling defaults to 1.5, which is quite suitable for dynamics.
		\item \textbf{“image”:} The command prints an image file.
		The usual way for creating such a command would seem to be using the other script, \texttt{buildglyphimages.py}, but this way you can also create commands using image files you obtained/created from any other source than LilyPond.
		You can even use this to create non-musical commands that print images and profit from \lilyglyphs*' infrastructure (like the automatic scaling with text size).
		You can use .pdf files (preferred) or .png or .jpg images for this purpose.
		You are encouraged to save these files to a dedicated folder different from the one where the LilyPond images are stored automatically.
		This folder has to be in view of \LaTeX, but if you use one inside the \lilyglyphs directory this isn't an issue.
		Of course such externally created image files aren't automatically tracked by Git and therefore aren't promoted to other users of the package.\\
		The “element” for an image driven command is the plain file name without extension or path.
	\end{itemize}
	\item \textsf{\textbf{scale / raise} (optional):} If one of these items is found they are set to be the new default value for the command's optional argument. 
	As described in \fref{subsec:BGI_preparing_input} the value is kept until the end of the file is reached or the value is set newly.
\end{itemize}

\noindent Here you can see two examples of command entries:
\begin{verbatim}
cmd=fermataDown
element=scripts.dfermata
comment=downward fermata
# type glyphname is implicitely used

cmd=rinforzando
element=rfz
comment=Rinforzando, to be kerned
type=dynamics
\end{verbatim}

\subsection{Running the script and Utilizing the Results}
\label{subsec:GGC_running_utilizing}

You can run the \texttt{/py/genGlyphCommands.py} script and simply pass the prepared input file as the first (and only) parameter.
The script will only look for the file in the \texttt{/stash\_new\_commands/emmentaler} directory.

The program parses the input file and creates a new file \texttt{/stash\_new\_commands/emmentaler/newGlyphCommands.tex} that is exactly like the one created by \texttt{buildglyphimages.py} (see \fref{subsec:BGI_utilizing_results}).
It doesn't create any intermediate files and doesn't invoke LilyPond, however.

You finish off the process exactly like with the image driven commands, so we don't have to repeat that here.

\pagebreak
\chapter{Reference of Predefined Commands}
\label{chap:list_predefined_commands}
The following sections document the predefined commands that already have been implemented.
They generally contain a list of all glyphs from the \emmentaler documentation, explanations on the use of the commands (if necessary) and a table listing the implemented commands.
Be aware that the lists and tables may be on different pages than the section header.
Sections whose implementations haven't started yet may be present or not, so if you are missing glyphs please refer to LilyPond's documentation.

The documentation contains information if the commands are available in starred and unstarred commands, and if they are based on image files or \emmentaler glyphs -- image commands are mentioned explicitely.

\section{Single Notes}
\label{sec:singlenotes}
Single notes may be one of the most frequently used glyphs.
They aren't present in \emmentaler*, so they all are realized using included pdf image files.
They have starred and unstarred versions.
See \fref{tab:singlenotes} for the available predefined commands.

\begin{reftable}{Single Notes}{singlenotes}
\crotchet & \cmd{crotchet}\\
\crotchetDown & \cmd{crotchetDown}\\
\crotchetDotted & \cmd{crotchetDotted}\\
\crotchetDottedDown & \cmd{crotchetDottedDown}\\
\crotchetDottedDouble & \cmd{crotchetDottedDouble}\\
\crotchetDottedDoubleDown & \cmd{crotchetDottedDoubleDown}\\
\halfNote & \cmd{halfNote}\\
\halfNoteDown & \cmd{halfNoteDown}\\
\halfNoteDotted & \cmd{halfNoteDotted}\\
\halfNoteDottedDown & \cmd{halfNoteDottedDown}\\
\halfNoteDottedDouble & \cmd{halfNoteDottedDouble}\\
\halfNoteDottedDoubleDown & \cmd{halfNoteDottedDoubleDown}\\
\quaver & \cmd{quaver}\\
\quaverDown & \cmd{quaverDown}\\
\quaverDotted & \cmd{quaverDotted}\\
\quaverDottedDown & \cmd{quaverDottedDown}\\
\quaverDottedDouble & \cmd{quaverDottedDouble}\\
\quaverDottedDoubleDown & \cmd{quaverDottedDoubleDown}\\
\semiquaver & \cmd{semiquaver}\\
\semiquaverDown & \cmd{semiquaverDown}\\
\semiquaverDotted & \cmd{semiquaverDotted}\\
\semiquaverDottedDown & \cmd{semiquaverDottedDown}\\
\semiquaverDottedDouble & \cmd{semiquaverDottedDouble}\\
\semiquaverDottedDoubleDown & \cmd{semiquaverDottedDoubleDown}\\
\end{reftable}


\section{Beamed notes}
\label{sec:beamed_notes}

We will only provide a few complex symbols like beamed notes for default use.
Of course one could have the wish for indefinite variations like  notes with variable beam slope.
But as long as it isn't possible to make this parametrical and draw them directly from within \lilyglyphs it is probably a good idea to stick to a few basic commands.
However it would be quite simple to create “plugin modules” that provide a series of related symbols, and maybe we will think about such an extension.
For now see \fref{tab:beamedNotes} for the implemented commands. They are available in starred and unstarred versions.

\begin{reftable}{Beamed Notes}{beamedNotes}
\twoBeamedQuavers & \cmd{twoBeamedQuavers}\\
\end{reftable}


\section{Clefs}
\label{sec:clefs}
%\glyphlist{\emmentaler Clef glyphs}{clefs}{lily-76dbcd67}
Some of the clef glyphs are among the few that are too large by default. 
You couldn't use a G clef within continuous text without severely \clefG damaging line spacing. 
But if you scale them to a size that doesn't disturb line spacing, they look quite disproportionate, especially when combined with other elements: \mbox{ \clefCInline \natural.}
Therefore we provide the clefs in two forms, a standard form which can be problematic in continuous text, and an -Inline version which looks somewhat funny but can be used within the line.
The clef commands are available in the starred and in the unstarred form.
See \fref{tab:clefs} for the available predefined commands.

\begin{reftable}{Clefs}{clefs}
\clefGInline & \cmd{clefG}, \cmd{clefGInline} & clefs.G\\
\clefFInline & \cmd{clefF}, \cmd{clefFInline} & clefs.F\\
\clefCInline & \cmd{clefC}, \cmd{clefCInline} & clefs.C\\
\end{reftable}

\section{Time Signatures}
\label{sec:timesignatures}
%\glyphlist{\emmentaler Number glyphs}{timesignatures}{lily-8b332c94}

\emmentaler provides two “real” glyphs for time signatures, the \lilyTimeC and the \lilyTimeCHalf*.
The commands \cmd{lilyTimeC} and \cmd{lilyTimeCHalf} have starred and unstarred versions. 

Numerical (single and compound) time signatures can be printed using \cmd{lilyTimeSignature}: \lilyTimeSignature{4}{4}.
This command expects two mandatory arguments: numerator and denominator.
They are treated as \cmd{lilyText}, so you can easily write compound time signatures like \cmd{lilyTimeSignature\{4 + 7\}\{8\}}: \lilyTimeSignature{4 + 7}{8}.

\cmd{lilyTimeSignature} doesn't need a starred version as it behaves as you would expect: It doesn't print a trailing space but respects any whitespace after the closing bracket.

\begin{reftable}{Time Signatures}{timesignatures}
\lilyTimeC & \cmd{lilyTimeC} & timesig.C44\\
\lilyTimeCHalf & \cmd{lilyTimeCHalf} & timesig.C22\\
\lilyTimeSignature{7}{8} & \cmd{lilyTimeSignature\{7\}\{8\}}\\
\lilyTimeSignature{3 + 4}{4 + 8} & \cmd{lilyTimeSignature\{3 + 4\}\{4 + 8\}}\\
\end{reftable}

\paragraph*{Known issues and warnings:} \cmd{lilyTimeSignature} also expects the optional argument as the other commands, but it doesn't understand the \texttt{raise} option correctly.
The box with the time signature is vertically centered so it should generally be OK, but if you for some reason have to change its vertical position you should surround it manually by a \cmd{raisebox}.

\section{Numbers}
\label{sec:numbers}
%\glyphlist{\emmentaler Number glyphs}{numbers}{lily-8d8bb8a3}
Numbers can be entered with the already known \cmd{lilyText} command. 
Access through the glyph names is possible but not necessary.
Therefore we don't provide predefined commands for them.
There are all ten numbers available. 
With the default scaling of 1.0 they generally fit as lowercase letters like \lilyText{0 1 2 3 4 5 6 7 8 9} \cmd{lilyText\{0 1 2 3 4 5 6 7 8 9\}}. 
For Uppercase letters you can start trying a scaling of 1.3. A future version of the package will provide convenience functions with default scalings for upper/lowercase letters, fingerings, figured bass numbers, time signature numbers etc.

A special case are four glyphs that are grouped among the numbers:\\
\lilyText[scale=1.5]{+~-~.~,} (plus, hyphen, fullstop and comma). \\
These are also accessible through \cmd{lilyText}, the example in the previous sentence being written as \cmd{lilyText[scale=1.5]\{+~-~.~,\}}.

\section{Accidentals}
\label{sec:accidentals}
%\glyphlist{\emmentaler Accidental glyphs}{accidentals}{lily-5b13ce04}
%For the complete list of glyphs see \fref{fig:accidentals}.

The \cmd{natural} \natural*, the \cmd{flat} \flat and the \cmd{sharp} \sharp replace the respective commands from standard \LaTeX. 
Please note that all the accidentals are designed at the same scaling in order to allow a uniform appearance. You will however have to check if they don't affect an even line spacing.

Accidentals are available in starred and unstarred versions.

See \fref{tab:accidentals} for the list of implemented commands. 

\begin{reftable}{Accidentals}{accidentals}
\natural & \cmd{natural} & accidentals.natural\\
\midrule
\sharp & \cmd{sharp} & accidentals.sharp\\
\sharpArrowup & \cmd{sharpArrowup} & accidentals.sharp.arrowup\\
\sharpArrowdown & \cmd{sharpArrowdown} & accidentals.sharp.arrowdown\\
\sharpArrowboth & \cmd{sharpArrowboth} & accidentals.sharp.arrowboth\\
\sharpSlashslashStem & \cmd{sharpSlashslashStem} & accidentals.sharp.slashslash.stem\\
\sharpSlashslashslashStemstem & \cmd{sharpSlashslashslashStemstem} & accidentals.sharp.slashslashslash.stemstem\\
\sharpSlashslashslashStem & \cmd{sharpSlashslashslashStem} & accidentals.sharp.slashslashslash.stem\\
\sharpSlashslashStemstemstem & \cmd{sharpSlashslashStemstemstem} & accidentals.sharp.slashslash.stemstemstem\\
\doublesharp & \cmd{doublesharp} & accidentals.doublesharp\\
\midrule
\flat & \cmd{flat} & accidentals.flat\\
\flatflat & \cmd{flatflat} & accidentals.flatflat\\
\end{reftable}

\section{Rests}
\label{sec:rests}
%\glyphlist{\emmentaler Rest glyphs}{rests}{lily-40869867}

Rests are usually available in starred and unstarred versions.
See \fref{tab:rests} for the implemented commands.

For more information on how to use \cmd{lilyPrintMoreDots} to produce multiply dotted rests please see \fref{sec:dotted_symbols}.

\begin{reftable}{Rests}{rests}
\wholeNoteRest & \cmd{wholeNoteRest} & Whole Note Rest\\
\wholeNoteRestDotted & \cmd{wholeNoteRestDotted} & DottedWhole Note Rest\\
\midrule
\halfNoteRest & \cmd{halfNoteRest} & Half Note Rest\\
\halfNoteRestDotted & \cmd{halfNoteRestDotted} & Dotted Half Note Rest\\
	\halfNoteRestDotted*\lilyPrintMoreDots & 
	\cmd{halfNoteRestDotted*}\cmd{lilyPrintMoreDots} &
	Example of Double Dotted Rest\\
\midrule
\crotchetRest & \cmd{crotchetRest} & Crotchet Rest\\
\crotchetRestDotted & \cmd{crotchetRestDotted} & Dotted Crotchet Rest\\
\midrule
\quaverRest & \cmd{quaverRest} & Quaver Rest\\
\quaverRestDotted & \cmd{quaverRestDotted} & Dotted Quaver Rest\\
\midrule
\semiquaverRest & \cmd{semiquaverRest} & Semiquaver Rest\\
\semiquaverRestDotted & \cmd{semiquaverRestDotted} & Dotted Semiquaver Rest\\
\end{reftable}

\section{Dynamic Text}
\label{sec:dynamic_text}
As explained earlier the Dynamic Letters can be accessed through \cmd{lilyText} without providing glyph names or numbers as argument. 
For the available letters see \ref{tab:singleDynLetters}. 
As a convenience there is a predefined command \cmd{lilyDynamics}, which is just a wrapper around \cmd{lilyText} that sets the \texttt{Scale} argument to a default value of 1.5.

\begin{reftable}{Single Dynamics Letters}{singleDynLetters}
\lilyDynamics{f} & \cmd{lilyDynamics\{f\}} & forte\\
\lilyDynamics{p} & \cmd{lilyDynamics\{p\}} & piano\\
\lilyDynamics{m} & \cmd{lilyDynamics\{m\}} & mezzo-\\
\lilyDynamics{r} & \cmd{lilyDynamics\{r\}} & rin-\\
\lilyDynamics{s} & \cmd{lilyDynamics\{s\}} & s-\\
\lilyDynamics{z} & \cmd{lilyDynamics\{z\}} & -z\\
\end{reftable}

These Letters can be combined to make complex Dynamics. 
\lilyglyphs doesn't provide a full set of predefined commands as they can easily be entered as single strings to \cmd{lilyDynamics}, like \cmd{lilyDynamics\{sffzrmp\}}, resulting in \lilyDynamics{sffzrmp}. 
In this specific situation you could enter a small horizontal space between the \lilyDynamics{z} and the \lilyDynamics{r} -- but as this combination wouldn't occur in real life, we don't need to demonstrate it here.
There are a few predefinied commands (see \fref{tab:combinedDynLetters}) that are only used for the combination of letters that need some special “kerning” attention. These commands internally use \cmd{lilyDynamics} with its default scaling. These predefined commands need starred and unstarred versions again, as \LaTeX would ignore spaces after the unstarred command.

\begin{reftable}{Combined Dynamics Expressions}{combinedDynLetters}
\lilyRF* & \cmd{lilyRF}, \cmd{lilyRF*} & rinforzando\\
\lilyRFZ* & \cmd{lilyRFZ}, \cmd{lilyRFZ*} & rinforzando (alternative)\\

\end{reftable}


\section{Graphical Dynamic Symbols}
\label{sec:graphical_dynamic_symbols}

Graphical dynamic symbols are realized by including image files and available as starred and unstarred versions.
See \fref{tab:dynamicSigns} for the implemented commands.

\begin{reftable}{Dynamics Signs}{dynamicSigns}
\crescHairpin & \cmd{crescHairpin}\\
\decrescHairpin & \cmd{decrescHairpin}\\
\end{reftable}

\section{Scripts}
\label{sec:scripts}
%\glyphlist{\emmentaler Script glyphs (part 1)}{scriptsI}{lily-20e8632d-1}
%\glyphlist{\emmentaler Script glyphs (part 2)}{scriptsII}{lily-20e8632d-2}

Script implementation has just begun. For the implemented glyphs \fref{tab:scripts}.

You may find it strange that script glyphs sometimes seem to print too far to the left, clashing with the preceding text.
This is due to the fact that in musical engraving scripts are centered relative to the note they belong to.
Therefore you often have to add extra space before the glyph if you access them directly.
The predefined commands should of course have this already built in.

\begin{reftable}{Scripts}{scripts}
\fermata & \cmd{fermata} & Fermata\\
\end{reftable}

\pagebreak
\appendix
\small
\chapter{Appendix: Licenses}
\label{chap:licenses}

\input{licenses.inp}

\end{document}
