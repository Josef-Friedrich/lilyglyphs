\documentclass{article}
\usepackage{xltxtra}
\title{The \textbackslash lilyglyphs Package}
\author{Urs Liska}
\usepackage{lilyglyphs}

% TODO make this prettier, preferrably using musical glyphs 
% like e.g. quarter notes for the 'l's and maybe a
% flipped \flat for the 'p'
\newcommand{\lilyglyphs}{\texttt{\textbf{lilyglyphs }}}

% Format lilyglyphs commands and save typing of the backslash
\newcommand*{\cmd}[1]{\texttt{\textbackslash #1}}

% Common table format for reference tables
% TODO make this prettier
\newcommand{\tmpCaption}{} % Necessary to pass caption to the end definition
\newcommand{\tmpLabel}{}
\newenvironment{reftable}[2]
	{%
		\renewcommand{\tmpCaption}{#1}
		\renewcommand{\tmpLabel}{#2}
		\begin{table}[ht]
		\begin{center}
		\begin{tabular}[t]{lll}
		\hline
		&\\
	}
	{%
		&\\
		\hline
		\end{tabular}
		\caption{\tmpCaption}
		\label{table:\tmpLabel}
		\end{center}
		\end{table}
	}


\begin{document}
\maketitle
\tableofcontents
\section{Introduction}
The package \lilyglyphs is made for authors who want to include single musical elements in their \LaTeX{} texts. This is not meant to be used for musical examples, for which one would use packages like \texttt{musixtex} or the LilyPond software together with lilypond-book\footnote{http://www.lilypond.org}. Here we are talking about entering items like \lilyRFZ* into continuous text, within the paragraphs.

For this purpose the package provides commands to include glyphs from LilyPond's Emmentaler OpenType font. For the implications of this statement see the next section "Requirements".

The \lilyglyphs package was initiated and is maintained by Urs Liska, and is hosted as a git repository on GitHub\footnote{http://www.github.com/uliska/lilyglyphs -- Contact: git@ursliska.de}

\section{Requirements}
\lilyglyphs achieves its goal through accessing OpenType fonts provided by the LilyPond notation software. These fonts are accessed through the \texttt{fontspec} package, and \lilyglyphs relies on a \LaTeX{} distribution supporting \texttt{fontspec}. It was written using \XeLaTeX, but should also work with LuaLaTeX -- but this hasn't been tested.

Of course the font has to be accessible to \texttt{fontspec}. Please refer to the \texttt{fontspec} documentation on how to make fonts visible to it. 

\textit{On a Linux system it may be enough to copy the font files to your home directory's .fonts directory in or a new subdirectory, possibly running \texttt{fc-cache} afterwards to update the fontconfig cache. The font files are located in the usr/share/lilypond/current/fonts/otf subdirectory of your LilyPond installation. An .otf file is included in the distribution archive of \lilyglyphs.}



\section{Usage}
\subsection{General usage}
For activating the capabilities of \lilyglyphs just write \cmd{usepackage\{lilyglyphs\}} in the preamble of your document. Then you can use the commands defined by the package or access glyphs directly through one of the generic access commands.

The predefined commands are generally available in two forms, starred and unstarred (e.\,g. \cmd{flat} and \cmd{flat*}). The unstarred commands provide the glyphs without trailing space while the starred ones append a single space after the glyph. The latter are more convenient to use in continuous text while the plain commands can be used either before punctuations or to create combined glyphs (e.\,g. \lilyS\lilyF\lilyF\lilyZ).

Glyphs are scaled to fit normal text fonts, and the scaling automatically follows the scaling of the text. This also means they are influenced by \LaTeX 's text size commands, so you can write something like \{\cmd{large \textbackslash flat\}} to create a larger {\large \flat} glyph than the normal \flat* one or even something really \cmd{huge} {\huge \lilySFZ} -- which will probably wreck your text layout. If you need more fine-grained control you can use the generic commands and set the scaling individually.

You can always combine commands (or glyphs created through generic access as shown in the next section) to make new commands. If you find yourself doing this regularely then please provide useful combinations as feature requests on our issue tracker. Or even better, join us and participate yourself.



\subsection{Generic access}
\lilyglyphs uses its generic commands \cmd{lilyGlyph} and \cmd{lilyGlyphByNumber} to access the glyphs. If you need a glyph that isn't covered yet by \lilyglyphs you can access it directly through these commands. Besides that they provide control over the size of the glyph.

Both commands expect two arguments. The first -- optional -- one is a scaling factor for the glyph because most of the glyphs won't fit the surrounding text font. You may start with a value of one but usually you will have to increase the value to something between 1.2 and 1.5 in order to create glyphs suiting the continuous text. Of course you can also use this argument to create glyphs of arbitrary size. 

The second argument is the glyph to be selected. \cmd{lilyGlyph} expects the OpenType glyph name, surrounded by double quotation marks. You can look up the glyph names in the Appendix of LilyPond's \emph{Notation Reference} \footnote{http://www.lilypond.org/doc/v2.15/Documentation/notation/the-feta-font.html}. \cmd{lilyGlyphByNumber} expects the Unicode code of the glyph. You will generally not want to use this as the code positions aren't guaranteed to stay the same with new versions of the fonts. There may be some uses for numerical access however.

\subparagraph*{Example:}
The coda sign isn't implemented yet, so you can create it with \cmd{lilyGlyph\{"scripts.coda"\}}: \lilyGlyph{"scripts.coda"}. If you want you can supply the optional \texttt{Scale} argument to make the glyph somewhat larger: ~\lilyGlyph[1.3]{"scripts.coda"}\\
\cmd{lilyGlyph[1.3]\{"scripts.coda"\}}.

As you can see the glyph is -- as most \emph{Emmentaler} glyphs are -- placed too low, so you have to place it in a  \cmd{raisebox}: \raisebox{0.4em}{\lilyGlyph[1.3]{"scripts.coda"}}, finding out the offset value by trial-and-error: \cmd{raisebox\{0.4em\}\{\textbackslash lilyGlyph\{"scripts.coda"\}\}}. The syntax is still not fixed yet, because in version 0.1 of \lilyglyphs the \cmd{raisebox} will be directly incorporated in the \cmd{lilyGlyph} command, with the raise amount being an optional argument.


\subsection{Available commands}

\subsubsection{Time Signatures}
\emph{Emmentaler} provides two 'real' glyphs for time signatures, the \lilyTimeC and the \lilyTimeCHalf. The commands \cmd{lilyTimeC} and \cmd{lilyTimeCHalf} don't have starred versions as the glyphs have their own space built in already. You can see this also at the end of the preceding sentence, where the period is too far away from the glyph.

More time signatures have to be constructed from numbers, which is a desideratum for the package.

\begin{reftable}{Time Signatures}{timesignatures}
\lilyTimeC & \cmd{lilyTimeC}\\
\lilyTimeCHalf & \cmd{lilyTimeCHalf}\\
\end{reftable}


\subsubsection{Numbers}
Numbers are entered with the command \cmd{lilyNumber}, giving the number as the argument. There are numbers from 0 to 9, not only from 1 to 5, so they can be used for fingerings or anything else Example: \cmd{lilyNumber*\{1\}} \lilyNumber*{1} with following text.


\subsubsection{Accidentals}
The package starts up with only a few of the standard accidentals of traditional Western music. More should be added soon. We have the \cmd{natural} \natural, the \cmd{flat} \flat* and the \cmd{sharp} \sharp* -- which replace the respective commands from standard \LaTeX. Additionally there are the double versions, \cmd{flatflat} \flatflat* and \cmd{doublesharp} \doublesharp. See table~\ref{table:accidentals} on page~\pageref{table:accidentals}. There you will also find the glyphnames.

Accidentals are available in starred and unstarred versions.

\begin{reftable}{Accidentals}{accidentals}
\natural & \cmd{natural} & "accidentals.natural"\\
\sharp & \cmd{sharp} & "accidentals.sharp"\\
\doublesharp & \cmd{doublesharp} & "accidentals.doublesharp"\\
\flat & \cmd{flat} & "accidentals.flat"\\
\flatflat & \cmd{flatflat} & "accidentals.flatflat"\\
\end{reftable}



\subsubsection{Dynamic Text}
First we have the basic Dynamic Letters \lilyP, \lilyF* -- and their 'modifiers'
\lilyM, \lilyS* and \lilyZ. See table~\ref{table:singleDynLetters} on page~\pageref{table:singleDynLetters}.

\begin{reftable}{Single Dynamic Letters}{singleDynLetters}
\lilyF* & \cmd{lilyF} & forte\\
\lilyP* & \cmd{lilyP} & piano\\
\lilyM* & \cmd{lilyM} & mezzo-\\
\lilyR* & \cmd{lilyR} & rin-\\
\lilyS* & \cmd{lilyS} & s-\\
\lilyZ* & \cmd{lilyZ} & -z\\
\end{reftable}

These Letters can be combined to make complex Dynamics. \lilyglyphs provides a set of predefined commands (see table~\ref{table:combinedDynLetters} on page~\pageref{table:combinedDynLetters}) but you can also create custom Dynamics. Keep in mind to use the unstarred versions of the commands for this, otherwise you'll have space between the letters. The predefined commands are also available in starred and unstarred versions, providing or not space after the whole group.
If you find yourself creating commands yourself often, then feel free to suggest them as enhancement requests. Or contact us and participate directly.

\begin{reftable}{Combined Dynamic Letters}{combinedDynLetters}
\lilyPPPP* & \cmd{lilyPPPP} & pianissimo-pianissimo\\
\lilyPPP* & \cmd{lilyPPP} & piano-pianissimo\\
\lilyPP* & \cmd{lilyPP} & pianissimo\\
\lilyMF* & \cmd{lilyMF} & mezzoforte\\
\lilyFF* & \cmd{lilyFF} & fortissimo\\
\lilyFFF* & \cmd{lilyFFF} & forte-fortissimo\\
\lilyFFFF* & \cmd{lilyFFFF} & fortissimo-fortissimo\\

\lilySF* & \cmd{lilySF} & sforzato\\
\lilySFZ* & \cmd{lilySFZ} & sforzato (alternative)\\
\lilyRF* & \cmd{lilyRF} & rinforzando\\
\lilyRFZ* & \cmd{lilyRFZ} & rinforzando (alternative)\\

\end{reftable}


\end{document}
