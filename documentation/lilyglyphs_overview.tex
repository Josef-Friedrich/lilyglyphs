\documentclass[DIV=13]{scrartcl}
\usepackage{lilyglyphsStyle}


%%%%%%%%%%%%%%%%%%%%
%% Hauptschriften %%
\setmainfont[%
	Numbers=OldStyle,
	Numbers=Proportional]	
	{Minion Pro}
\setsansfont[%
	Numbers=OldStyle,
	Numbers=Proportional]
	{Cronos Pro}

\begin{document}

\section*{\centering\Huge\lilyglyphs[scale=1.5]}
\begin{flushright}
Urs Liska, November 2012
\end{flushright}
Lange Zeit hatte ich mich um das Problem herumlaviert, Notenzeichen in Texten verwenden zu müssen.
Aus Anlass eines zu setzenden Revisionsberichts zu einer Notenausgabe habe ich mich der Fragestellung nun noch einmal angenommen und eine elegante und umfassende Lösung gefunden: ein Paket, das die Möglichkeit eröffnet, beliebige Notationselemente im Fließtext von \LaTeX-Dokumenten zu verwenden.

\texttt{lilyglyphs} ist ein \LaTeX-Paket, das

\end{document}